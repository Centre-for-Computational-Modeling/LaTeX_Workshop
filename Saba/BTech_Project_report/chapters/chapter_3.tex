

\chapter{Literature Survey}
A literature review is an overview of the previously published works on a topic. The term can refer to a full scholarly paper or a section of a scholarly work such as a book, or an article. Either way, a literature review is supposed to provide the researcher/author and the audiences with a general image of the existing knowledge on the topic under question. A good literature review can ensure that a proper research question has been asked and a proper theoretical framework and/or research methodology have been chosen. To be precise, a literature review serves to situate the current study within the body of the relevant literature and to provide context for the reader. In such case, the review usually precedes the methodology and results sections of the work.

Producing a literature review is often a part of graduate and post-graduate student work, including in the preparation of a thesis, dissertation, or a journal article. Literature reviews are also common in a research proposal or prospectus (the document that is approved before a student formally begins a dissertation or thesis).

A literature review can be a type of review article. In this sense, a literature review is a scholarly paper that presents the current knowledge including substantive findings as well as theoretical and methodological contributions to a particular topic. Literature reviews are secondary sources and do not report new or original experimental work. Most often associated with academic-oriented literature, such reviews are found in academic journals and are not to be confused with book reviews, which may also appear in the same publication. Literature reviews are a basis for research in nearly every academic field.

\section{Types}

Since the concept of a systematic review was formalized (codified) in the 1970s, a basic division among types of reviews is the dichotomy of narrative reviews versus systematic reviews. The term literature review without further specification still refers (even now, by convention) to a narrative review.

The main types of narrative reviews are evaluative, exploratory, and instrumental.

A fourth type of review, the systematic review, also reviews the literature (the scientific literature), but because the term literature review conventionally refers to narrative reviews, the usage for referring to it is "systematic review". A systematic review is focused on a specific research question, trying to identify, appraise, select, and synthesize all high-quality research evidence and arguments relevant to that question. A meta-analysis is typically a systematic review using statistical methods to effectively combine the data used on all selected studies to produce a more reliable result.[3]

