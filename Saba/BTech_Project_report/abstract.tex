
\newpage
\addcontentsline{toc}{chapter}{\MakeUppercase{Abstract}}
\thispagestyle{empty}
%
\begin{spacing}{1.5}
%
\chapter*{Abstract}
%%
%% Here enter abstract. No figures, sketches, tables shall be here.
%%

From Wikipedia, the free encyclopedia.

An abstract is a brief summary of a research article, thesis, review, conference proceeding, or any in-depth analysis of a particular subject and is often used to help the reader quickly ascertain the paper's purpose. When used, an abstract always appears at the beginning of a manuscript or typescript, acting as the point-of-entry for any given academic paper or patent application. Abstracting and indexing services for various academic disciplines are aimed at compiling a body of literature for that particular subject.Academic literature uses the abstract to succinctly communicate complex research. An abstract may act as a stand-alone entity instead of a full paper. As such, an abstract is used by many organizations as the basis for selecting research that is proposed for presentation in the form of a poster, platform/oral presentation or workshop presentation at an academic conference. Most bibliographic databases only index abstracts rather than providing the entire text of the paper. Full texts of scientific papers must often be purchased because of copyright and/or publisher fees and therefore the abstract is a significant selling point for the reprint or electronic form of the full text.The abstract can convey the main results and conclusions of a scientific article but the full text article must be consulted for details of the methodology, the full experimental results, and a critical discussion of the interpretations and conclusions. Abstracts are occasionally inconsistent with full reports. This has the potential to mislead clinicians who rely solely on the information present in the abstract without consulting the full report.






%%
\end{spacing}